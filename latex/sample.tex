\documentclass[12pt]{article}

\usepackage[T1]{fontenc}
\usepackage[utf8]{inputenc}
\usepackage[french]{babel}
\usepackage[left=3cm,right=3cm,top=2.5cm,bottom=2.5cm]{geometry}

\usepackage{hyperref}
\usepackage{color}
\usepackage{graphicx}
\usepackage{listings}

\title{Rendu \LaTeX}
\author{de Muret de Labouret Michel}

% Cela permet de faire des retours a la ligne avant les debuts de sous-sections en priorité
\let\LaTeXStandardSubsection\subsection
\makeatletter
\renewcommand{\subsection}[1]{\filbreak\LaTeXStandardSubsection{#1}}
\makeatother

% Definitions pour le package color
\definecolor{red}{rgb}{1,0,0}
\definecolor{green}{rgb}{0,1,0}
\definecolor{blue}{rgb}{0,0,1}
\definecolor{darkWhite}{rgb}{0.94,0.94,0.94}
\definecolor{gray}{rgb}{0.5,0.5,0.5}

% Parametrage pour le package listing
\lstset{
  aboveskip=3mm,
  belowskip=-2mm,
  backgroundcolor=\color{darkWhite},
  basicstyle=\footnotesize,
  breakatwhitespace=false,
  breaklines=true,
  captionpos=b,
  commentstyle=\color{red},
  deletekeywords={...},
  escapeinside={\%*}{*)},
  extendedchars=true,
  framexleftmargin=16pt,
  framextopmargin=3pt,
  framexbottommargin=6pt,
  frame=tb,
  keepspaces=true,
  keywordstyle=\color{blue},
  language=C,
  literate=
  {²}{{\textsuperscript{2}}}1
  {⁴}{{\textsuperscript{4}}}1
  {⁶}{{\textsuperscript{6}}}1
  {⁸}{{\textsuperscript{8}}}1
  {€}{{\euro{}}}1
  {é}{{\'e}}1
  {è}{{\`{e}}}1
  {ê}{{\^{e}}}1
  {ë}{{\¨{e}}}1
  {É}{{\'{E}}}1
  {Ê}{{\^{E}}}1
  {û}{{\^{u}}}1
  {ù}{{\`{u}}}1
  {â}{{\^{a}}}1
  {à}{{\`{a}}}1
  {á}{{\'{a}}}1
  {ã}{{\~{a}}}1
  {Á}{{\'{A}}}1
  {Â}{{\^{A}}}1
  {Ã}{{\~{A}}}1
  {ç}{{\c{c}}}1
  {Ç}{{\c{C}}}1
  {õ}{{\~{o}}}1
  {ó}{{\'{o}}}1
  {ô}{{\^{o}}}1
  {Õ}{{\~{O}}}1
  {Ó}{{\'{O}}}1
  {Ô}{{\^{O}}}1
  {î}{{\^{i}}}1
  {Î}{{\^{I}}}1
  {í}{{\'{i}}}1
  {Í}{{\~{Í}}}1,
  morekeywords={*,...},
  numbers=left,
  numbersep=10pt,
  numberstyle=\tiny\color{black},
  rulecolor=\color{black},
  showspaces=false,
  showstringspaces=false,
  showtabs=false,
  stepnumber=1,
  stringstyle=\color{gray},
  tabsize=4,
  title=\lstname,
}



\begin{document}

\maketitle

\begin{abstract}
Cet article a pour but de montrer ce que je sais faire en Latex.
\end{abstract}

\newpage

\tableofcontents

\newpage

\filbreak
\section{Les textes}

\subsection{Gras italique}
Il est possible de mettre certaines parties du texte en \textbf{gras} ou-bien en \textit{italique}.

\subsection{Police}
\textsf{J'aime bien cette police.}
\textsc{Je n'aime pas trop celle là.}

\subsection{Couleurs}
Enfin, utiliser de la couleur permet de \textcolor{red}{faire ressortir} des passages importants de la phrase. Mais il 
ne faut pas trop en abuser \fcolorbox{blue}{red}{au} \fcolorbox{red}{blue}{risque} \fcolorbox{blue}{red}{de} \fcolorbox{red}{blue}{perdre} le lecteur.

\subsection{Bibliographie}
Vu que je ne sais pas trop quand citer un document pour la bibliographie, je vais le faire maintenant \cite{preparata-1988}.


\filbreak
\section{Jeu Tents}

\subsection{Tableau}
\label{Tableau}

Voici à quoi ressemble le jeu par défaut du jeu tents tel que defini sur le site de référence \textcolor{blue}{\url{https://www.chiark.greenend.org.uk/~sgtatham/puzzles/js/tents.html}}:

\medskip

\begin{center}
\begin{tabular}{ c | c  c  c  c  c  c  c  c | }		
        & 4 & 0 & 1 & 2 & 1 & 1 & 2 & 1 \\
      \hline
     3 &   &   &   &   & T & T &   &   \\
     0 & T &   &   &   &   &   &   & T \\
     4 &   &   &   &   & T &   &   &   \\
     0 & T &   &   &   &   & T &   &   \\
     4 &   & T &   &   &   &   &   &   \\
     0 & T &   &   &   & T &   & T &   \\
     1 &   &   &   &   &   &   &   &   \\
     0 & T &   &   &   &   &   &   &   \\
    \hline  
    \end{tabular}
    
  \end{center}

\subsection{Image}

Voici à quoi ressemble notre interface graphique du jeu tents pour le jeu par défaut
presenté dans la partie\hyperref[Tableau]{ \textcolor{blue}{~\ref{Tableau}} }:

\medskip

\includegraphics[height=10cm]{tents_default.png}


\filbreak
\section{Mathématiques}

\LaTeX est particulièrement adapté à la rédaction de textes mathématiques: 


\[(a+b)^n = \sum_{k=0}^{n} (^n_k ) a^{n-k} b^k \]

\[ z_1 = \sqrt{x^2_1 + y^2_1} \]

\[ T(I) = \sum_{i=0}^{I} \frac{h(i)}{M} \]

\[ X = \frac{\int_{\lambda_{min}}^{\lambda_{max}} E(\lambda)S(\lambda)\overline{x}(\lambda)d\lambda}{\int_{\lambda_{min}}^{\lambda_{max}}S(\lambda)\overline{y}(\lambda)d\lambda} \]


\filbreak
\section{Informatique}

Ce code en forme de donut génère l'animation d'un donut en art ASCII  :

\begin{lstlisting}
             k;double sin()
         ,cos();main(){float A=
       0,B=0,i,j,z[1760];char b[
     1760];printf("\x1b[2J");for(;;
  ){memset(b,32,1760);memset(z,0,7040)
  ;for(j=0;6.28>j;j+=0.07)for(i=0;6.28
 >i;i+=0.02){float c=sin(i),d=cos(j),e=
 sin(A),f=sin(j),g=cos(A),h=d+2,D=1/(c*
 h*e+f*g+5),l=cos      (i),m=cos(B),n=s\
in(B),t=c*h*g-f*        e;int x=40+30*D*
(l*h*m-t*n),y=            12+15*D*(l*h*n
+t*m),o=x+80*y,          N=8*((f*e-c*d*g
 )*m-c*d*e-f*g-l        *d*n);if(22>y&&
 y>0&&x>0&&80>x&&D>z[o]){z[o]=D;;;b[o]=
 ".,-~:;=!*#$@"[N>0?N:0];}}/*#****!!-*/
  printf("\x1b[H");for(k=0;1761>k;k++)
   putchar(k%80?b[k]:10);A+=0.04;B+=
     0.02;}}/*****####*******!!=;:~
       ~::==!!!**********!!!==::-
         .,~~;;;========;;;:~-.
             ..,--------,*/
\end{lstlisting}

Comme on peut le voir sur le gif suivant : \textcolor{blue}{\url{https://hackaday.com/wp-content/uploads/2020/07/spinning-donut-featured.gif?w=800}}

\filbreak
Et voici une version équivalente sans obfuscation :

\begin{lstlisting}
  int k;
  double sin();
  double cos();
  main() {
    float A=0;
    float B=0;
    float i;
    float j;
    float z[1760];
    char  b[1760];
    printf("\x1b[2J");
    for (;;) {
      memset(b, 32, 1760);
      memset(z, 0, 7040);
      for (j = 0; 6.28 > j; j += 0.07) {
        for (i = 0; 6.28 > i; i += 0.02) {
          float c = sin(i);
          float d = cos(j);
          float e = sin(A);
          float f = sin(j);
          float g = cos(A);
          float h = d + 2;
          float D = 1 / (c * h * e + f * g + 5);
          float l = cos(i);
          float m = cos(B);
          float n = sin(B);
          float t = c * h * g - f * e;
   
          int x = 40 + 30 * D * (l * h * m - t * n);
          int y = 12 + 15 * D * (l * h * n + t * m);
          int o = x + 80 * y;
          int N = 8 * ((f * e - c * d * g) * m - c * d * e - f * g - l * d * n);
   
          if (22 > y && y > 0 && x > 0 && 80 > x && D > z[o]) {
            z[o] = D;
            b[o] = ".,-~:;=!*#$@"[N > 0 ? N : 0];
          }
        }
      }
      printf("\x1b[H");
      for (k = 0; 1761 > k; k++) {
        putchar(k % 80 ? b[k] : 10);
      }
      A += 0.04;
      B += 0.02;
    }
  }
\end{lstlisting}

\newpage

\bibliographystyle{plain}
\bibliography{biblio} % fichier biblio.bib

\end{document}

