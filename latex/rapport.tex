\documentclass{article}

\usepackage[french]{babel}
\usepackage[utf8]{inputenc}
%\usepackage{lmodern}
\usepackage[T1]{fontenc}
\usepackage{hyperref}
\usepackage{times}
\usepackage{graphicx}
\usepackage{csquotes}
\usepackage{amssymb}

\title{Pavages et pavabilité dans le plan}
\author{A. HIPPOLYTE, R. CHAVAGNAC, M. DE MURET DE LABOURET}

\begin{document}

\maketitle

\tableofcontents

\section{Introduction}

\subsection{Pavage du plan}

Un pavage P dans un espace E est un ensemble fini de n éléments {p1,p2,...,pn} appartenant à E et sous-espace de P. Chaque element est compacts et d’interieur non vide.
P a une frontière F (points au contact de P et de son complementaire(ensemble des elements de E qui n'appartienent pas à P)) et un intérieur (emsemble des points de P qui n'appartiennent pas à F).
On dit que P est fermé si F appartient à P.
La reunion des sous-ensemble {p1,p2,...,pn} est egale à P. C'est une partition du plan (euclidien, hyperbolique (1) …).
On peut aussi paver dans un espace de dimension supérieur au plan mais nous allons nous concentrer sur l’espace à deux dimensions.
Chaque élément de P possède au moins un côté connexe a un autre.
Il peut y avoir plusieurs éléments de forme différentes pour paver un plan.

\subsection{Pavage régulier}

Lorsque les même éléments sont représentés plusieurs fois, on dit que le pavage est regulier (irrégulier si non).
Les sous-espaces de P sont image l’un de l’autre par une isométrie(3) du plan (translation, rotation,symetrie centrale, symétrie axiale, symétrie glissée…).
Pour tout couple (p1,p2) de P régulier, il existe une isometrie du pavage f tel que f(p1)=p2.


\subsection{Pavage périodique / aperiodique}

\textbf{\textsc{Périodique}}

Le pavage périodique est composé d'elements répétés sur une portion regulière du plan.
On peut le retrouver sur nos carrelages ou sur les dessins de M.C Escher (pavage hyperbolique).
Ce type de pavage est connu depuis l’antiquité comme motif décoratif.
Le mathematicien russe Fedorov a montré que seulement 17 types de pavages periodiques existaient dans le plan.
Deux pavages sont considérés de meme type s’ils sont invariant par le meme groupe d’isométrie(3).

\textbf{\textsc{Apériodique}}

Le pavage apériodique est composé d'ensemble d'élements répétés apériodiquement.
Le pavage de Penrose (4) est l'exemple le plus connu.



\section{Pavage du plan par des dominos}

Il n’existe à ce jour, aucun algorithme qui dit systématiquement si une portion de plan est pavable par des polygones.
On dit que le problème est indécidable (6).

Dans cette partie, nous allons alors nous concentrer sur le pavage par dominos, le domino 2x1 et 1x2.
Soit P le pavage construit par les dominos dans Z2. P est une portion finie de plan délimitée par des côtés de longueur entière parallèles aux axes du plan.
On considere aussi que ce sous espace de Z2 n’a pas de trous. Ce type de pavage est régulier par isométrie du plan et apériodique.
Les dominos ne doivent pas se chevaucher et sont connexes à au moins un autre domino.

\subsection{Polymonio}

Un domino est un polymonio(5) appartenant à Z2 composé de 2 carrés. Il n’a qu’une forme libre possible et deux formes fixées : 2x1 et 1x2.
Un triomino est composé de 3 carrés et a deux formes libres possibles.
Ainsi de suite, un decamino (10 carrés) a 4655 formes libres différentes

\subsection{Nombre de pavage dans un rectangle mxn}
\textbf{\textsc{2xn}}

Soit Fn le nombre de pavage possible.
Le premier cas est le rectangle 2x1, il y a bien sûr qu’un pavage possible : F1 = 1
Le deuxieme cas est un rectangle 2x2, où deux pavages sont possibles deux dominos en long ou deux dominos en large.
Dans un cas général, si on ajoute au rectangle 2xn un rectangle sur le coté de longueur 2, le nombre Fn de pavage augmente de 1.
Mais on peut aussi ajouter deux rectangles en long au rectangle 2x(n-1) sur le coté de longueur 2. Dans ce cas la, le nombre Fn-1 de pavage augmente de 2.
Ainsi on a tous les pavages du rectangle 2x(n+1) et Fn+1 = Fn + Fn-1.
Ce sont les nombres de Fibonacci : 1, 2, 3, 5, 8, 13, 21, 34, 55…
Le rectangle de dimension 2x10 a 89 pavages possibles.

\subsection{Algorithme de Thurston}


\textbf{\textsc{definitions}}

(1)	Espace hyperbolique : espace qui ne vérifie pas le 5eme postulat d’Euclide (2) . Il existe une infinité de droites différentes parallèle à une même droite.
(2)	Prenons un point  P extérieur à une droite d1, il n’existe qu’une droite d2 parallèle à d1 qui passe par P.
(3)	Isometrie.
(4)	Pavage de Penrose: pavage qui répond à plusieurs règles précises Il existe 4 pavages de Penrose différents :
Pentagonal ( avec des pentagones, losanges, pentagrammes et des portions de pentagramme).
Cerfs-volant et fléchettes ( avec deux quadrilatères, l'un convexe, l'autre concave )
Losanges ( avec deux sortes de losanges, fins et gros )
Triangle d’or ( avec des triangles isocèles dont les cotés ont des longueurs proportionnelles au nombre d’or )
(5)	Polyomino : réunion connexe de carrés unitaires (ex: formes de Tetris) chaque carré a au moins un coté connexe a un autre, ils peuvent être en 3D)
Il existe 2 groupes de polyomino : a forme fixée et a forme libre (qui peut subir une rotation ).
(6)	Probleme indecidable : demontré par Alan Turing en 1936, est un probleme qui ne peut pas etre resolu algorithmiquement en un temps toujours fini. Ce probeleme est vrai sur les machines de Turing.


$\forall \chi \in \mathbb{Z}, \hat{u}(\chi) = \sum_{x=0}^{N-1} e^{-2i\Pi\frac{\chi x}{N}}$

% 

\bibliographystyle{plain}
\bibliography{biblio} % fichier biblio.bib
\end{document}

