\documentclass{article}

\usepackage[french]{babel}
\usepackage[utf8]{inputenc}
%\usepackage{lmodern}
\usepackage[T1]{fontenc}
\usepackage{hyperref}
\usepackage{times}
\usepackage{graphicx}
\usepackage{csquotes}
\usepackage{amssymb}


\title{Pavages et pavabilité dans le plan}
\author{A. HIPPOLYTE, R. CHAVAGNAC, M. DE MURET DE LABOURET}

\begin{document}

\maketitle

\begin{abstract}
    Voici le rapport de notre projet tutoré
\end{abstract}

\tableofcontents

\section{Première partie}

\begin{displayquote}
    \emph{"Physics is the universe’s operating system", Steven R Garman}
\end{displayquote}

\subsection{Première sous-partie}

\textbf{\textsc{Petit titre}}

Pavage et pavabilité dans le plan
Pavage du plan : ensemble de formes géométriques dont l’union forme une partition du plan euclidien. Chaque forme possède au moins un côté connexe a une autre.
Il est possible d’assembler plusieurs formes différentes pour paver un plan.
Il existe deux types de pavage, le pavage périodique (motif répété sur une grille régulière), que l’on peut retrouver sur nos carrelages ou sur les dessins de M.C Escher.
Et le pavage apériodique comme le pavage de Penrose
Pavage de Penrose: pavage qui répond à plusieurs règles précises
Il existe 4 pavages de Penrose différents :
Pentagonal ( avec des pentagones, losanges, pentagrammes et des portions de pentagramme).
Cerfs-volants et fléchettes ( avec deux quadrilatères, l'un convexe, l'autre concave )
Losanges ( avec deux sortes de losanges, fins et gros )
Triangle d’or ( avec des triangles isocèles dont les cotés ont des longueurs proportionnelles au nombre d’or )
Dans cette partie, nous allons nous concentrer sur le pavage par dominos.
Chacun est l’image d’un autre par une isométrie du plan (translation, rotation,symetrie centrale, symétrie axiale, symétrie glissée…)
Un domino est un polymonio en 2D (réunion connexe de carrés unitaires (ex: formes de Tetris) chaque carré a au moins un côté connexe a un autre)
Il existe 2 groupes de polyomino: a forme fixée et a forme libre (qui peut subir une rotation )
Dans le cas du pavage, les polyominos sont à forme libre puisque leur pavabilité de dépend pas de leur sens.
Un domino est un polyomino composé de 2 carrés. Il n’a qu’une forme libre possible.
Un triomino est composé de 3 carrés et a deux formes libres possibles.
Ainsi de suite, un decamino (10 carrés) a 4655 formes libres différentes


\begin{center}
    \cite{einstein}

\end{center}


\subsection{Deuxième sous-partie}
\textbf{\textsc{Become a homebody}}

Lorem ipsum dolor sit amet, consectetur adipiscing elit. Sed rhoncus libero ac nisl pulvinar,
vitae placerat massa dignissim. Donec metus leo, convallis in elit nec, lacinia porta purus.
Maecenas ut leo eget neque vestibulum eleifend at id ex.

\section{Deuxième partie}

\subsection{Première sous-partie}
\textbf{\textsc{Petit titre}}

Lorem ipsum dolor sit amet, consectetur adipiscing elit. Sed rhoncus libero ac nisl pulvinar,
vitae placerat massa dignissim. Donec metus leo, convallis in elit nec, lacinia porta purus.
Maecenas ut leo eget neque vestibulum eleifend at id ex.

\subsection{Deuxième sous-partie}
\textbf{\textsc{Petit titre}}

Lorem ipsum dolor sit amet, consectetur adipiscing elit. Sed rhoncus libero ac nisl pulvinar,
vitae placerat massa dignissim. Donec metus leo, convallis in elit nec, lacinia porta purus.
Maecenas ut leo eget neque vestibulum eleifend at id ex. \newline\newline
Lorem ipsum dolor sit amet\newline


$\forall \chi \in \mathbb{Z}, \hat{u}(\chi) = \sum_{x=0}^{N-1} e^{-2i\Pi\frac{\chi x}{N}}$

% 

\bibliographystyle{plain}
\bibliography{biblio} % fichier biblio.bib
\end{document}

